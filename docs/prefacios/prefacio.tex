\thispagestyle{empty}

\begin{center}
{\large\bfseries Sistema multiplataforma para la búsqueda y gestión de pisos compartidos }\\
\end{center}
\begin{center}
Alonso Doña Martínez\\
\end{center}

%\vspace{0.7cm}

\vspace{0.5cm}
\noindent\textbf{Palabras clave}: \textit{Software libre}, \textit{Tecnología aplicada a la vivienda}, \textit{Co-Housing}, \textit{Gestión de pisos compartidos}, \textit{Desarrollo FullStack}
\vspace{0.7cm}

\noindent\textbf{Resumen}\\

Actualmente, uno de los principales problemas de la sociedad es la constante escalada de los precios en el ámbito de la vivienda, un desafío que afecta tanto a jóvenes como a adultos.

Una de las soluciones que está ganando terreno para mitigar los costes es la búsqueda de viviendas compartidas. Sin embargo, a pesar del contexto tecnológico actual, donde existen múltiples plataformas y portales para la búsqueda de viviendas y la organización de tareas, todos presentan limitaciones al no centrarse en ofrecer soluciones eficientes que pongan al usuario en el centro. En lugar de tratar la vivienda solo como un espacio, estas plataformas deberían considerar también los lazos y relaciones entre las personas que la habitan, brindándoles herramientas para la gestión y organización de tareas y promoviendo una convivencia agradable.

Esta \textit{plataforma}\footnote{El repositorio del proyecto está disponible en: 
\href{https://github.com/alonsodm12/TFG_COHOUSING}{Repositorio en GitHub}} nace con el objetivo de cambiar el enfoque tradicional de los pisos compartidos, creando un entorno agradable que cubra las necesidades e intereses de todos los miembros de la vivienda.



\cleardoublepage
% textidote: ignore begin %
\begin{center}
	{\large\bfseries Multiplatform system for the search and management of shared housing}\\
\end{center}
\begin{center}
	Alonso Doña Martínez\\
\end{center}
\vspace{0.5cm}
\noindent\textbf{Keywords}: \textit{Free software}, \textit{Technology applied to housing}, \textit{Co-Housing}, \textit{Shared housing management}, \textit{FullStack Development}
\vspace{0.7cm}

\noindent\textbf{Abstract}\\

Currently, one of the main problems in society is the constant rise in housing prices, a challenge that affects both young people and adults.

One of the solutions that is gaining traction to reduce costs is the search for 
shared housing. However, despite the current technological landscape, where 
multiple platforms and portals exist for finding housing and organizing tasks, 
they all have limitations as they do not focus on providing efficient and 
convenient solutions that put the user at the center. Instead of treating 
housing merely as a space, these platforms should also consider the bonds and 
relationships between the people who live there, offering them tools for task 
management and organization while promoting a pleasant coexistence.

This \textit{platform}\footnote{The repository for this project is available on GitHub: 
\href{https://github.com/alonsodm12/TFG_COHOUSING}{Github Repository}} was created with the goal of changing the traditional approach to 
shared housing, fostering an environment that aligns with the needs and 
interests of all household members.

% textidote: ignore end %

\cleardoublepage
% textidote: ignore begin %
\begin{center}
	{\large\bfseries Agradecimientos}\\
\end{center}
En primer lugar me gustaría expresar mi agradecimiento a mi tutor de TFG, Luis López Escudero, por su ayuda, compromiso y guía durante los meses de desarrollo de este proyecto.

Agradecer al personal docente de la Universidad de Granada por la dedicación y enseñanza durante estos años, claves en mi formación tanto profesional como personal.

A mi familia, por su apoyo y respaldo constante, por estar conmigo y ayudarme siempre.

A mi hermano Rafa, por ser una inspiración constante durante estos años de carrera.

A mis amigos y compañeros, por realizar este camino conmigo y estar siempre en los momentos más duros, por las noches de estudio y por las risas.

A Carlos, por tu compañerismo y amistad tanto en los momentos difíciles como bonitos de esta etapa. 

Finalmente, me gustaría agradecer a todas las personas que han permitido que el desarrollo de este proyecto haya sido posible.
