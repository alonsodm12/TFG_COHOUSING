\chapter{Introducción} \label{introduccion}

\section{Contexto}
En la actualidad, el acceso a la vivienda se ha convertido en un desafío tanto para jóvenes 
como adultos, especialmente en grandes ciudades donde los precios de compra y alquiler siguen 
aumentando drásticamente. 

En este contexto, ha surgido una nueva alternativa de convivencia, el 
\emph{co-housing} \cite{cohousing}, una solución innovadora que combina el acceso a una vivienda digna con el 
aprovechamiento de recursos y espacios compartidos, fomentando la colaboración y la gestión 
conjunta. Sin embargo, este modelo suele aplicarse en infraestructuras diseñadas específicamente 
para ello, lo que implica un alto coste inicial y una planificación a largo plazo.

Dado que los pisos son la forma de vivienda más extendida en las ciudades, adaptar los principios 
del \emph{co-housing} —como la búsqueda de personas afines y la vida en comunidad— a esta modalidad permitirá ofrecer una solución que mejore la calidad de vida de los individuos. Además, 
permitirá optimizar la organización de tareas y mejorar la convivencia entre los integrantes.

A pesar de sus evidentes beneficios, la gestión de pisos compartidos siguiendo estos principios presenta ciertos 
desafíos, como la organización y reparto de tareas o la búsqueda en función de la afinidad. En este sentido, la 
tecnología puede desempeñar un papel crucial en la resolución 
de estos problemas.

Este trabajo tiene como objetivo desarrollar una plataforma de creación y gestión de comunidades 
en pisos compartidos, permitiendo a los usuarios administrar estas viviendas de forma sencilla y 
moderna, fomentando un modelo de convivencia colaborativo.

\section{Motivación}
El crecimiento exponencial de las tendencias de convivencia en comunidad y el incremento del precio de la vivienda ha generado la necesidad de una plataforma tecnológica que facilite la creación, organización y administración de estos entornos en la vivienda más común: los pisos.

A pesar de la existencia de plataformas que ayudan en la búsqueda de viviendas o en la organización de tareas diarias, muchas de ellas carecen de toda la funcionalidad necesaria para aplicar los principios del \emph{co-housing}. En general, estas plataformas suelen abordar solo uno de los principios fundamentales de este tipo de convivencia, sin proporcionar una solución completa.

Teniendo esto en cuenta, la motivación de este proyecto radica en las siguientes cuestiones:

\begin{itemize}
    \item \textbf{Falta de plataformas especializadas}: Las plataformas actuales no están diseñadas para gestionar de manera completa la convivencia en pisos compartidos bajo los principios del \emph{co-housing}. Esto dificulta la implementación efectiva de este modelo de convivencia.
    
    \item \textbf{Plataforma moderna e intuitiva}: Se busca ofrecer una aplicación robusta, con una interfaz moderna y fácil de usar, que permita a los usuarios interactuar de forma sencilla y eficiente.
    
    \item \textbf{Solución a un problema actual}: El acceso a la vivienda es un desafío creciente. Según el informe del Parlamente Europeo \href{https://www.europarl.europa.eu/topics/es/article/20241014STO24542/el-aumento-del-coste-de-la-vivienda-en-la-union-europea?utm_source=chatgpt.com}{(Europarl, 2024)}, el coste de la vivienda ha aumentado significativamente en la gran mayoría de países de la Zona Euro \cite{DatosMacro}, lo que ha llevado a muchas personas a optar por viviendas compartidas como alternativa asequible. Sin embargo, la falta de herramientas adecuadas para gestionar esta convivencia puede generar conflictos y reducir la calidad de vida de los residentes.
\end{itemize}

Con base en estos puntos, este trabajo propone el desarrollo de una plataforma tecnológica que afronte estas necesidades. De esta manera, no solo se atiende una demanda real del mercado, sino que también se aplican conocimientos avanzados de ingeniería de software, siguiendo estándares de desarrollo profesional.

\section{Objetivos}
El objetivo principal de este proyecto es la creación de un sistema multiplataforma que permita la gestión de pisos compartidos, ofreciendo una respuesta moderna que aborde los principios de búsqueda por afinidad y gestión eficiente de tareas. Este objetivo se ha articulado en torno a la consecución de los siguientes objetivos:

\begin{itemize}
    \item \textbf{OE1:} Evaluar la arquitectura y tecnologías más convenientes para el problema.
    \item \textbf{OE2:} Estudiar la metodología a seguir y realizar una planificación en torno a ella.
    \item \textbf{OE3:} Utilizar \emph{GitHub} \cite{GitHub} como control de versiones e integrarlo en todo el ciclo de vida de desarrollo.
    \item \textbf{OE4:} Desarrollar una interfaz moderna e intuitiva para el usuario con tiempos de carga reducidos.
    \item \textbf{OE5:} Desarrollar la gestión de tareas que permita al usuario ver qué tiene que hacer en cada momento y qué le toca realizar a sus compañeros.
    \item \textbf{OE6:} Permitir la visualización del estado de desarrollo de las tareas.
    \item \textbf{OE7:} Desarrollar un algoritmo de recomendación que permita poner en contacto usuarios afines.
    \item \textbf{OE8:} Aprender sobre los frameworks de desarrollo \emph{Spring Boot} \cite{SpringBoot} y \emph{React} \cite{React} y ponerlos en práctica en el desarrollo del proyecto.
\end{itemize}

\section{Estructura de la memoria}
Durante el primer capítulo se han presentado tanto el contexto como la motivación para este trabajo, así como los objetivos principales que permitirán abordar el problema. El resto del contenido de la memoria se ha estructurado de la siguiente forma:

\begin{itemize}
    \item En el capítulo 2 se ha realizado una descripción del estado del arte en relación a las principales tecnologías y técnicas presentes en el mercado sobre gestión de viviendas compartidas, algoritmos de recomendación y desarrollo de aplicaciones multiplataforma.
    \item En el capítulo 3 se va a proceder a realizar ...
    \item El último capítulo aborda las conclusiones y los trabajos futuros que pueden surgir a partir de este TFG.
    \item Para finalizar, se muestran las referencias bibliográficas usadas durante la realización de este trabajo.
\end{itemize}